\documentclass{report}

\usepackage[finnish]{babel}
\usepackage[utf8]{inputenc}
\usepackage{graphicx}
\usepackage[ampersand]{easylist}

\begin{document}

\section*{Johdanto}
\subsection*{Järjestelmän tarkoitus}
Järjestelmä tarjoaa käyttäjilleen mahdollisuuden tallentaa askareitaan muistiin ja listata niitä niiden tärkeyden mukaan. Se siis auttaa muistamaan päivän askareet, jolloin ne tulee varmasti suoritettua.

\subsection*{Toteutus-/toimintaympäristö}
Työ toteutetaan laitoksen users-palvelimelle käyttäen PHP-kieltä. Sovellus tulee käyttämään PostgreSQL-tietokantapalvelinta.

\section*{Yleiskuva järjestelmästä}
\subsection*{Käyttötapauskaavio}
\includegraphics[scale=0.66]{kaavio}

\subsection*{Käyttäjäryhmät}
\subsubsection*{Jokamies}
Jokamiehellä tarkoitetaan ketä tahansa, joka Internetin välityksellä tulee sovelluksen etusivulle. 

Myös sidosryhmä "Käyttäjä" kuuluu tähän sidosryhmään.

\subsubsection*{Käyttäjä}
Käyttäjä on henkilö, joka on rekisteröitynyt sovellukseen sekä kirjautunut sisään.

\subsection*{Käyttötapauskuvaukset}
\ListProperties(Hang=true, Hide=50, Progressive=3ex, Style*=$\bullet$ , Style2*=--)
\subsubsection*{Jokamiehen käyttötapaukset}
\begin{easylist}
& Rekisteröityminen
& Sivuston avaaminen
\end{easylist}

\subsubsection*{Käyttäjän käyttötapaukset}
\begin{easylist}
& Askareiden hallinoiminen
&& Lisäys, muokkaus, luokkittelu ja poistaminen
&& Tärkeyttäminen
&&& Asetetaan askareelle tärkeysaste
& Tärkeyasteiden hallinnoiminen
&& Lisäys ja poisto
& Luokkien hallinnoiminen
&& Lisäys ja poisto
& Kirjautuminen 
\end{easylist}

\end{document}