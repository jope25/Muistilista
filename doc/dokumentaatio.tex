\documentclass{report}

\usepackage[finnish]{babel}
\usepackage[utf8]{inputenc}
\usepackage{graphicx}
\usepackage[ampersand]{easylist}
\usepackage{hyperref}

\begin{document}

\section*{Johdanto}
\subsection*{Järjestelmän tarkoitus}
Järjestelmä tarjoaa käyttäjilleen mahdollisuuden tallentaa askareitaan muistiin ja listata niitä niiden tärkeyden mukaan. Se siis auttaa muistamaan päivän askareet, jolloin ne tulee varmasti suoritettua.

\subsection*{Toteutus-/toimintaympäristö}
Työ toteutetaan laitoksen users-palvelimelle käyttäen PHP-kieltä. Sovellus tulee käyttämään PostgreSQL-tietokantapalvelinta.

\section*{Yleiskuva järjestelmästä}
\subsection*{Käyttötapauskaavio}
\includegraphics[scale=0.66]{ktkaavio}

\subsection*{Käyttäjäryhmät}
\subsubsection*{Jokamies}
Jokamiehellä tarkoitetaan ketä tahansa, joka Internetin välityksellä tulee sovelluksen etusivulle. 

Myös sidosryhmä "Käyttäjä" kuuluu tähän sidosryhmään.

\subsubsection*{Käyttäjä}
Käyttäjä on henkilö, joka on rekisteröitynyt sovellukseen sekä kirjautunut sisään.

\subsection*{Käyttötapauskuvaukset}
\ListProperties(Hang=true, Hide=50, Progressive=3ex, Style*=$\bullet$ , Style2*=--)
\subsubsection*{Jokamiehen käyttötapaukset}
\begin{easylist}
& Rekisteröityminen
& Sivuston avaaminen
\end{easylist}

\subsubsection*{Käyttäjän käyttötapaukset}
\begin{easylist}
& Askareiden hallinoiminen
&& Lisäys, muokkaus, luokkittelu ja poistaminen
&& Tärkeyttäminen
&&& Asetetaan askareelle tärkeysaste
& Tärkeyasteiden hallinnoiminen
&& Lisäys ja poisto
& Luokkien hallinnoiminen
&& Lisäys ja poisto
& Kirjautuminen 
\end{easylist}

\section*{Järjestelmän tietosisältö}
\subsection*{Käsitekaavio}
\includegraphics[scale=0.75]{kasitekaavio}

\subsection*{Tietokohteet}
\subsubsection*{Käyttäjä}
\begin{tabular}{ | l | l | l | p{7.5cm} |} \hline
\textbf{Attribuutti} & \textbf{Arvojoukko} & \textbf{Kuvailu} \\ \hline
Nimi & Merkkijono, max. 25 merkkiä & Käyttäjän nimi sovelluksessa \\ \hline
Salasana & Merkkijono, max. 50 merkkiä & Merkkijono, jolla käyttäjä varmennetaan \\ \hline
\end{tabular}
\\ \\ Käyttäjällä voi olla useita askareita, tärkeysasteita ja luokkia, joiden kaikkien olemmassaolo riippuu käyttäjästä.

\subsubsection*{Askare}
\begin{tabular}{ | l | l | l | p{7.5cm} |} \hline
\textbf{Attribuutti} & \textbf{Arvojoukko} & \textbf{Kuvailu} \\ \hline
Käyttäjä & Luku & Käyttäjän id \\ \hline
Tärkeysaste & Luku & Tärkeysasteen id \\ \hline
Valmis & boolean & Onko askare tehty \\ \hline
Nimi & Merkkijono, max. 25 merkkiä & Askareen nimi \\ \hline
Lisätty & Date & Päivämäärä, jolloin askare lisätty \\ \hline
Lisätieto & Merkkijono, max. 500 & Lisätietoa askareesta ja tarkempaa kuvailua \\ \hline
\end{tabular}
\\ \\ Askare on yhden käyttäjän hallinnoima ja sillä voi olla yksi tärkeysaste. Lisäksi sillä voi olla useita askareluokkia. 

\subsubsection*{Luokka}
\begin{tabular}{ | l | l | l | p{7.5cm} |} \hline
\textbf{Attribuutti} & \textbf{Arvojoukko} & \textbf{Kuvailu} \\ \hline
Käyttäjä & Luku & Käyttäjän id \\ \hline
Nimi & Merkkijono, max. 25 merkkiä & Luokan nimi \\ \hline
Lisätieto & Merkkijono, max. 500 & Lisätietoa luokasta \\ \hline
\end{tabular}
\\ \\ Luokka on yhden käyttäjän hallinnoima. Luokalla voi olla useita askareluokkia.
\subsubsection*{Tärkeysaste}
\begin{tabular}{ | l | l | l | p{7.5cm} |} \hline
\textbf{Attribuutti} & \textbf{Arvojoukko} & \textbf{Kuvailu} \\ \hline
Käyttäjä & Luku & Käyttäjän id \\ \hline
Nimi & Merkkijono, max. 25 merkkiä & Tärkeysasteen nimi \\ \hline
Tärkeys & Kokonaisluku, 1-5 & Kuinka tärkeä aste on, 5 on tärkein \\ \hline
Lisätieto & Merkkijono, max. 500 & Lisätietoa tärkeysasteesta \\ \hline
\end{tabular}
\\ \\ Tärkeysaste on yhden käyttäjän hallinnoima. Lisäksi tärkeysasteeseen voi liittyä useita askareita.

\subsubsection*{Askareluokka}
\begin{tabular}{ | l | l | l | p{7.5cm} |} \hline
\textbf{Attribuutti} & \textbf{Arvojoukko} & \textbf{Kuvailu} \\ \hline
Askare & Luku & Askare id \\ \hline
Luokka & Luku & Luokka id \\ \hline
\end{tabular}
\\ \\ Askareluokka on liitostauluna luokan ja askareen välillä.

\section*{Relaatiotietokantakaavio}
\includegraphics[scale=0.6]{rtkk}

\section*{Käynnistys- / käyttöohje}
\noindent Kirjautumissivu on toistaiseksi sovelluksen etusivu. \\ Osoite: \href{http://jpetro.users.cs.helsinki.fi/muistilista/kirjautuminen}{http://jpetro.users.cs.helsinki.fi/muistilista/kirjautuminen} \\
Käyttäjätunnus on "Kimi" ja salasana "Kimi123".

\end{document}